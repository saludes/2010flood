%% LyX 2.0.0 created this file.  For more info, see http://www.lyx.org/.
%% Do not edit unless you really know what you are doing.
%\documentclass[english]{article}
%\usepackage[T1]{fontenc}
%\usepackage[latin9]{inputenc}
%\usepackage{babel}

\section{Introduction}

The initial data is the forecasted avenue hydrogram (the graph of
the flow as a function of time) computed from observations at a point
upstream of the control point. Two parameters are relevant: the \emph{maximal
flow} and the \emph{total volume} of the avenue.

We assume that the water height at the control point is a increasing
function of the flow $Q$ at that point.

Let us considere three states related to the flow $Q$:\global\long\def\Qmin{Q_{\mbox{min}}}
\global\long\def\Qf{Q_{\mbox{f}}}
\global\long\def\Qmao{Q_{\mbox{mao}}}
\global\long\def\Qmax{Q_{\mbox{max}}}
\global\long\def\Qi{Q_{\mbox{i}}}

\begin{itemize}
\item The \emph{steady state} corresponds to $Q<\Qmin$, where $\Qmin$
is the flow that brings the water at the control point high enough
for the gates to open. 
\item A\emph{ typical avenue} state corresponds to $\Qmin<Q<\Qmao$. The
flow $\Qmao$ is the maximum flow for a recurrent avenue, one that
happens every two or three years. 
\item The \emph{high avenue} state happens when $\Qmao<Q<\Qmax$, where
$\Qmax$ is the flow that makes the water level spill over the floodable
areas. 
\item Finally, when $Q>\Qmax$ 
\end{itemize}
For this the following control strategies are defined: 
\begin{itemize}
\item If the maximum forecast flow $\Qi$ is less than $\Qmao$, the gates
are not opened. 
\item When $\Qi>\Qmao$ and $\Qf<\Qmax$. In this case the goal is to minimize
the ??. To compute this, one cuts from the hydrogram an area equivalent
to the capacity of the floodable areas, to get a maxim plateau of
$\Qf$ flow. 
\end{itemize}

\section{Cutting the avenue}

We want to ensure $\Qf<\Qmax$ whenever possible, where $\Qmax$ is
the \emph{peak flow} that results of gates opening. 
\begin{itemize}
\item $Q(t)=\tilde{Q}(t)$ except for $t\in[t_{0},t_{1}]$ (the controlling
period) 
\item If $Q(t)<Q_{\mbox{min}}$, then $Q(t)=\tilde{Q}(t)$ (No gate opening
below this level) 
\item $\tilde{Q}(t)\leq Q(t)$, for all $t$; 
\item Denote by $V$ the total capacity of the floodable areas, Then
\begin{equation}
\int(Q(t)-\tilde{Q}(t))\, dt\leq V.\label{eq:capacity}
\end{equation}

\item If possible $Q_{\mbox{f}}=\sup\tilde{Q}<Q_{\mbox{max}}$ 
\end{itemize}
Define $R=Q-\tilde{Q}$. The previous conditions imply that $R$ is
positive, has compact support and $\int R\, dt\leq V$. Condition
?? implies that $R=0$.

An obvious choice for $\tilde{Q}$ is to pick $I=[a,b]$ such that
$\tilde{Q}(t)=Q(t)-Q_{\mbox{set}}$ for $t\in I$ and $Q(t)=\tilde{Q}(t)$
elsewhere. The capacity condition \eqref{eq:capacity} reads as 
\[
\int_{a}^{b}Q\, dt\leq V+Q_{\mbox{set}}(b-a)
\]
 and $\Qmax=Q_{\mbox{set}}$ in this case.

Let us call \emph{action support} the interval $I=[a,b]$ where $Q$
and $\tilde{Q}$ differ. To cut the most of the water we impose equality
on condition \eqref{eq:capacity} giving
\[
\int_{a}^{b}Q\, dt=\int_{a}^{b}\tilde{Q}\, dt+V
\]
it is also clear that to minimize $\max\tilde{Q}$ this function must
be constant $\tilde{Q}(t)=q$, for otherwise we could replace it for
a constant equal to its mean value on $I$. Therefore \eqref{eq:capacity}
reads:
\begin{equation}
\int_{a}^{b}Q\, dt=q(b-a)+V,\label{eq:prob1}
\end{equation}
and the problem is to find $a$, $b$ and fulfiling \eqref{eq:prob1}
and minimizing $\Qmax$. Add $Q(a)=Q(b)=q$ to \eqref{eq:prob1} and
we are set.


\subsection{The case of monotonic hydrograms}

When $Q$ has a single maximum at the time instant $a\leq t_{m}\leq b$
is it possible. Consider 
\[
W(t)=\int_{-\infty}^{t}Q\, d\tau.
\]
Notice that $W(t)$ is increasing and $Q(t)$ is increasing on the
subinterval $[a,t_{m}]$ and decreasing on $[t_{m},b]$ so we can
now take $W_{+}(q)=W(t_{+}^{-1}(q))$ defined on $[Q(a),\Qmax]$ and
$W_{-}(q)=W(t_{-}^{-1}(q))$ in the interval $[Q(b),\Qmax]$. Now
\[
f(q)=W_{-}(q)-W_{+}(q)=\int_{t_{+}}^{t_{-}}Q\, d\tau=V
\]


and $f(q)$ is a decreasing function whose range goes to zero. By
binary bisection it is very easy to find the point $q_{*}$ such that
$f(q_{*})=V$.


\section{Incorporating cost}

Let us now modify the assumption that the impact of flooding the urban
areas is constant by making it depend on the number of residents affected
by the flooding. Assume we have geographic data to describe $C(h)$,
the number of residents that live below heigh $h$. Let $C_{\mbox{f}}$
be the cost of flooding the fields, then the total cost will be 
\[
C=\left\{ \begin{array}{cc}
C(h) & \mbox{no flooding}\\
C_{\mbox{f}}+C(h(q_{*})) & \mbox{when flooding}
\end{array}\right.
\]
